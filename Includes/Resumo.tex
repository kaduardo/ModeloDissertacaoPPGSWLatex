\chapter*{}
% Resumo em língua vernácula
\begin{center}
	{\Large{\textbf{Definição de Perfis de Utilização e Predição de Problemas de Automóveis Baseados em Técnicas de Aprendizado de Máquina
}}}
\end{center}

\vspace{1cm}

\begin{flushright}
	Autor: Cephas Alves da Silveira Barreto\\
	Orientador: Prof. Dr. João Carlos Xavier Júnior
\end{flushright}

\vspace{1cm}

\begin{center}
	\Large{\textsc{\textbf{Resumo}}}
\end{center}

\noindent Considerando o incremento no número de automóveis nos grandes centros urbanos e os problemas que decorrem desse contexto, possuir ferramentas capazes de extrair informações a respeito do comportamento dos motoristas, torna-se uma importante estratégia para a diminuição dos problemas já existentes e também para a prevenção de problemas que possam vir a ocorrer neste domínio, como acidentes de trânsito, aumento da poluição, congestionamentos, distribuição dos tipos de transportes no espaço urbano e etc. Este artigo apresenta um estudo sobre as estratégias de definição de perfis de motoristas, mais especificamente as estratégias que utilizam técnicas de Aprendizado de Máquina para esse fim, propõe uma comparação entre diversas abordagens utilizadas para a definição de perfis de motoristas, enfatizando seus pontos fortes e também analisando possíveis limitações existentes.

\noindent\textit{Palavras-chave}: Machine Learning. Driving Behavior.




% resumo em inglês
%\begin{resumo}[Abstract]
% \begin{otherlanguage*}{english}
%   This is the english abstract.

%   \vspace{\onelineskip}
 
%   \noindent 
%   \textbf{Keywords}: latex. abntex. text editoration.
% \end{otherlanguage*}
%\end{resumo}
