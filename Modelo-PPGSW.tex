% Pre-ambulo
\documentclass[a4paper, 12pt]{abnt}

\usepackage[utf8]{inputenc}
\usepackage[brazil]{babel}
\usepackage[T1]{fontenc}
\usepackage{dsfont}
\usepackage{amssymb,amsmath}
\usepackage{multirow}
\usepackage[alf]{abntcite}
\usepackage[pdftex]{color, graphicx}
\usepackage{colortbl}
\usepackage{url}
\usepackage{abnt-alf}
\usepackage{abntcite}
\usepackage{capt-of}
\usepackage{eurosym}
%\usepackage{alg}
%\usepackage{hyperref}

%\usepackage{algorithm}
%\usepackage{algorithmic}
% Redefinicao de instrucoes
%\floatname{algorithm}{Algoritmo}
%\renewcommand{\algorithmicrequire}{\textbf{Entrada:}}
%\renewcommand{\algorithmicensure}{\textbf{Saída:}}
%\renewcommand{\algorithmicend}{\textbf{fim}}
%\renewcommand{\algorithmicif}{\textbf{se}}
%\renewcommand{\algorithmicthen}{\textbf{então}}
%\renewcommand{\algorithmicelse}{\textbf{senão}}
%\renewcommand{\algorithmicfor}{\textbf{para}}
%\renewcommand{\algorithmicforall}{\textbf{para todo}}
%\renewcommand{\algorithmicdo}{\textbf{faça}}
%\renewcommand{\algorithmicwhile}{\textbf{enquanto}}
%\renewcommand{\algorithmicloop}{\textbf{loop}}
%\renewcommand{\algorithmicrepeat}{\textbf{repetir}}
%\renewcommand{\algorithmicuntil}{\textbf{até que}}
%\renewcommand{\algorithmiccomment}[1]{\% #1}


% Definicao da lista de simbolos
% \simb[entrada na lista de simbolos]{simbolo}:
% Escreve o simbolo no texto e uma entrada na lista de simbolos.
% Se o parametro opcional e omitido, usa-se o parametro obrigatorio.
\newcommand{\simb}[2][]
{%
	\ifthenelse{\equal{#1}{}}
	{\addcontentsline{los}{simbolo}{#2}}
	{\addcontentsline{los}{simbolo}{#1}}#2
}
% Para aceitar comandos com @ (at) no nome
\makeatletter 
% \listadesimbolos: comando que imprime a lista de simbolos
\newcommand{\listadesimbolos}
	{
	\pretextualchapter{Lista de símbolos}
	{\setlength{\parindent}{0cm}
	\@starttoc{los}}
}
% Como a entrada sera impressa
\newcommand\l@simbolo[2]{\par #1}
\makeatother


% Definicao da lista de abreviaturas e siglas
% \abrv[entrada na lista de simbolos]{abreviatura}:
% Escreve a sigla/abreviatura no texto e uma entrada na lista de abreviaturas e siglas.
% Se o parametro opcional e omitido, usa-se o parametro obrigatorio.
\newcommand{\abrv}[2][]
{%
	\ifthenelse{\equal{#1}{}}
	{\addcontentsline{loab}{abreviatura}{#2}}
	{\addcontentsline{loab}{abreviatura}{#1}}#2
}
% Para aceitar comandos com @ (at) no nome
\makeatletter 
% \listadeabreviaturas: comando que imprime a lista de abreviaturas e siglas
\newcommand{\listadeabreviaturas}
{
	\pretextualchapter{Lista de abreviaturas e siglas}
	{\setlength{\parindent}{0cm}
	\@starttoc{loab}}
}
% Como a entrada sera impressa
\newcommand\l@abreviatura[2]{\par #1}
\makeatother


% \listofalgorithms: comando que imprime a lista de algoritmos
%\renewcommand{\listalgorithmname}{Lista de algoritmos}


% Hifenização de palavras feita de forma incorreta pelo LaTeX
% \hyphenation{PYTHON ou-tros}


% Inicio do documento
\begin{document}

	\frenchspacing
	
	% Capa (arquivo Includes/Capa.tex)
	% Capa
% Proteção externa do trabalho e sobre a qual se imprimem as informações indispensáveis 
% à sua identificação.

% Especificação da capa
\begin{titlingpage}
	\begin{center}
		% Cabeçalho (não deve ser modificado)
		% Contém o brasão da Universidade, o logotipo do Departamento, além dos dados
		% relacionados à vinculação do aluno (Universidade, Centro, Departamento e Curso)
		\begin{minipage}{2.2cm}
			\begin{center}
				\includegraphics[scale=1.2]{Imagens/Brasao-UFRN.jpg}
			\end{center}
		\end{minipage}
		\begin{minipage}{10.15cm}
			\begin{center}
			%	\begin{espacosimples}
					{\small \ \\
                       \textsc{Universidade Federal do Rio Grande do Norte}		   			\\
							  \textsc{Instituto Metrópole Digital}					\\
							  \textsc{Programa de Pós-graduação em Engenharia de Software}	\\
                       \textsc{Mestrado Profisional em Engenharia de Software}}   				\\
			%	\end{espacosimples}
			\end{center}
		\end{minipage}
		\begin{minipage}{2.2cm}
			\begin{center}
				\includegraphics[scale=0.1]{Imagens/Logotipo-IMD}
			\end{center}
		\end{minipage}
			
		\vspace{6cm}
						
		% Título do trabalho
		{\setlength{\baselineskip}%
		{1.3\baselineskip}
		{\LARGE \textbf{Definição de Perfis de Utilização e Predição de Problemas de Automóveis Baseados em Técnicas de Aprendizado de Máquina}}\par}
			
		\vspace{3cm}
			
		% Nome do aluno (autor)
		{\large \textbf{Cephas Alves da Silveira Barreto}}
						
		\vspace{6cm}
		
		% Local da instituição onde o trabalho deve ser apresentado e ano de entrega do mesmo
		Natal-RN\\Novembro de 2017
	\end{center}
\end{titlingpage}

	% Folha de rosto (arquivo Includes/FolhaRosto.tex)
	% Folha de rosto
% Contém os elementos essenciais à identificação do trabalho.

% Título, nome do aluno e respectivo orientador e filiação
\titulo{\Large{TITULO DO TRABALHO}}
\autor{Nome completo do autor}
\orientador[Orientador]{\par Nome completo do orientador e titulação}
\instituicao
{
   PPGSW -- Programa de Pós-graduação em Engenharia de Software\par
   IMD -- Instituto Metrópole Digital\par
   UFRN -- Universidade Federal do Rio Grande do Norte
}
	
% Natureza do trabalho (não deve ser modificada)
\comentario
{
	Dissertação de Mestrado  apresentada ao Programa de Pós-graduação em Engenharia de Software da Universidade Federal do Rio Grande do Norte como requisito parcial para a obtenção do grau de Mestre em Engenharia de Software.\bigskip\\
   \textit{Linha de pesquisa}:\\Uma das linhas do programa (Engenharia de sistemas Web, ou Jogos)
}
		
% Local e data
\local{Natal-RN}
\data{Mês e ano}

\folhaderosto	
	
	% Folha de aprovacao (arquivo Includes/FolhaAprovacao.tex)
	% Folha de aprovação
\begin{folhadeaprovacao}
	\setlength{\ABNTsignthickness}{0.4pt}
	\setlength{\ABNTsignwidth}{10cm}
	
	% Informações gerais acerca do trabalho 
	% (nome do autor, título, instituição à qual é submetido e natureza)
	\noindent 
	Dissertação de Mestrado sob o título \textit{TITULO DO TRABALHO} apresentada por \textbf{Nome do aluno} e aceita pelo Programa de Pós-graduação em Engenharia de Software da Universidade Federal do Rio Grande do Norte, sendo aprovada por todos os membros da banca examinadora abaixo especificada:
		
	% Membros da banca examinadora e respectivas filiações
	\assinatura
	{
		Prof. Dr. Fulano de tal,PhD   			                  \\
		{\small Presidente}											          \smallskip\\ 
		{\footnotesize
			IMD -- Instituto Metrópole Digital		   \\
		  	UFRN -- Universidade Federal do Rio Grande do Norte
		}
   }
      
   \assinatura
	{
      Nome completo do examinador e titulação   			                  \\
		{\small Examinador}											          \smallskip\\ 
		{\footnotesize
			Departamento		\\
		  	Universidade
		}
   }   
   
   \assinatura
	{
      Nome completo do examinador e titulação   			                  \\
		{\small Examinador}											          \smallskip\\ 
		{\footnotesize
			Departamento		\\
		  	Universidade
		}
	}
		
	\vfill
	
	\begin{center}
		Natal-RN, data da defesa (dia, mês e ano).
	\end{center}
\end{folhadeaprovacao}
	
	
	% Dedicatoria (arquivo Includes/Dedicatoria.tex)
	% Dedicatória

\chapter*{}
\vspace{15cm}
\begin{flushright}
	Homenagem que o autor presta a uma ou mais pessoas.
\end{flushright}
	
	% Agradecimentos (arquivo Includes/Agradecimentos.tex)
	% Agradecimentos

\chapter*{Agradecimentos}

Agradecimentos dirigidos àqueles que contribuíram de maneira relevante à elaboração do trabalho, sejam eles pessoas ou mesmo organizações.
   
   % Epigrafe (arquivo Includes/Epigrafe.tex)
	% Epígrafe (citação seguida de indicação de autoria)

\chapter*{}
\vspace{15cm}
\begin{flushright}
	\textit
	{
		Citação.
	}\medskip\\ 
	Autor.
\end{flushright}
	
	% Resumo em língua vernacula (arquivo Includes/Resumo.tex)
	\chapter*{}
% Resumo em língua vernácula
\begin{center}
	{\Large{\textbf{Título do trabalho}}}
\end{center}

\vspace{1cm}

\begin{flushright}
	Autor: Nome do aluno\\
	Orientador(a): Titulação e nome do(a) orientador(a)
\end{flushright}

\vspace{1cm}

\begin{center}
	\Large{\textsc{\textbf{Resumo}}}
\end{center}

\noindent O resumo deve apresentar de forma concisa os pontos relevantes de um texto, fornecendo uma visão rápida e clara do conteúdo e das conclusões do trabalho. O texto, redigido na forma impessoal do verbo, é constituído de uma seqüência de frases concisas e objetivas e não de uma simples enumeração de tópicos, não ultrapassando 500 palavras, seguido, logo abaixo, das palavras representativas do conteúdo do trabalho, isto é, palavras-chave e/ou descritores. Por fim, deve-se evitar, na redação do resumo, o uso de parágrafos (em geral resumos são escritos em parágrafo único), bem como de fórmulas, equações, diagramas e símbolos, optando-se, quando necessário, pela transcrição na forma extensa, além de não incluir citações bibliográficas.

\noindent\textit{Palavras-chave}: Palavra-chave 1, Palavra-chave 2, Palavra-chave 3.
	
	% Abstract, resumo em língua estrangeira (arquivo Include/Abstract.tex)
	\chapter*{}
% Resumo em língua estrangeira (em inglês Abstract, em espanhol Resumen, em francês Résumé)
\begin{center}
	{\Large{\textbf{Título do trabalho (em língua estrangeira)}}}
\end{center}

\vspace{1cm}

\begin{flushright}
	Author: Nome do aluno\\
	Supervisor: Titulação e nome do(a) orientador(a)
\end{flushright}

\vspace{1cm}

\begin{center}
	\Large{\textsc{\textbf{Abstract}}}
\end{center}

\noindent O resumo em língua estrangeira (em inglês \textit{Abstract}, em espanhol \textit{Resumen}, em francês \textit{Résumé}) é uma versão do resumo escrito na língua vernácula para idioma de divulgação internacional. Ele deve apresentar as mesmas características do anterior (incluindo as mesmas palavras, isto é, seu conteúdo não deve diferir do resumo anterior), bem como ser seguido das palavras representativas do conteúdo do trabalho, isto é, palavras-chave e/ou descritores, na língua estrangeira. Embora a especificação abaixo considere o inglês como língua estrangeira (o mais comum), não fica impedido a adoção de outras linguas (a exemplo de espanhol ou francês) para redação do resumo em língua estrangeira.


\noindent\textit{Keywords}: Keyword 1, Keyword 2, Keyword 3.
	
	% Lista de figuras
	\listoffigures

	% Lista de tabelas
	\listoftables
	
	% Lista de abreviaturas e siglas
	\listadeabreviaturas
	
	% Lista de símbolos
	\listadesimbolos
	
	% Lista de algoritmos (se houver)
	% Devem ser incluídos os pacotes algorithm e algorithmic
	% \listofalgorithms
	
	% Sumário
	\sumario

	% Parte central do trabalho, englobando os capítulos que constituem o mesmo
	% Os referidos capítulos devem ser organizados dentro do diretório "Capítulos"

	% Capitulo 1: Introdução (arquivo Includes/Introducao.tex)
	% Introdução
\chapter*{Introdução}

Segundo a Organização das Nações Unidas \cite{un2010state}, em 2050, 91,4\% da população da América Latina viverá em áreas urbanas. Assim como em outros países, se agravam no Brasil os problemas decorrentes do aumento da densidade demográfica nos centros urbanos. Poluição, epidemias, congestionamentos, violência e outros problemas têm se tornado uma questão urgente para os gestores dos espaços urbanos.  A poluição atmosférica, por exemplo, agravada com o aumento do número de veículos, é responsável por cerca de 20 mil óbitos por ano no Brasil \cite{arbex2012poluiccao}. O aumento do número de veículos é outro problema que vem tomando grandes proporções nas grandes cidades. Como pode ser visto na Figura \ref{fig:crescimentoveiculos}, esse problema tem se agravado especialmente a partir do ano de 2001 \cite{INCT2013evoluccao}.

\begin{figure}[ht]
\centering
\includegraphics[width=4.5in]{Imagens/carrosUp.png}. 
\caption{Crescimento de veículos no Brasil}
\label{fig:crescimentoveiculos}
\end{figure}

Nesse período, por exemplo, o crescimento do número de veículos nas regiões metropolitanas brasileiras foi de no mínimo 73,1\% (valor alcançado pelo Rio de Janeiro), e mais que dobrou em casos como Curitiba, Belém e outras, como é possível observar na Figura \ref{fig:crescimentocarros}. Isso indica que o crescimento do número de veículos está acontecendo com valores relativamente altos, e que é necessário ter formas de oferecer respaldo para as iniciativas que tentem lidar com o grande número de veículos e problemas decorrentes.

\begin{figure}[ht]
\centering
\includegraphics[width=4.5in]{Imagens/carrosUpBrasil.png}.  
\caption{Crescimento de carros nas regiões metropolitanas}
\label{fig:crescimentocarros}
\end{figure}

Ainda nas grandes regiões metropolitanas é possível observar que, em grande parte dos casos, o desenvolvimento urbano não acontece de forma sustentável \cite{rolnik2011crescimento} e não consegue acompanhar demandas urgentes como a necessidade de transporte para pessoas que vive nas metrópoles. Esse fato, aliado ao crescente número de pessoas, acaba transformando o ambiente urbano num plano ideal para o agravamento mais acelerado dos problemas locais. Parte desses problemas se relaciona de forma estreita com os veículos utilizados diariamente pelas pessoas.

Segundo a Wharton School (2013)\cite{WHARTON2013}, cerca de 6 a 7 bilhões de reais de prejuízo foram causados em todo o Brasil, apenas pelos acidentes envolvendo veículos no ano de 2013. Tais prejuízos financeiros se somam a tantos outros problemas, como por exemplo: os ambientais, devido à grande quantidade de poluentes lançados no ambiente; transtornos psicológicos nas pessoas envolvidas em acidentes com veículos; na produtividade, devido à grande quantidade de tempo desperdiçado com problemas de infraestrutura, trânsito ou nos próprios veículos, e etc. Com isso fica claro que o acúmulo de veículos traz consigo uma grande quantidade de problemas, dessa forma é importante pensar em abordagens que possam reduzir ou eliminar tais problemas.

Motivados pelos problemas relacionados acima e com o objetivo de oferecer uma forma de auxiliar no melhor uso de veículos, alguns trabalhos recentes \cite{zhang2016driver,amsalu2016driver,meseguer2015assessing,eren2012estimating}, têm focado em abordagens que permitem a criação de perfis automáticos de usuários (motoristas) de automóveis. Os resultados desses trabalhos possibilitam entender como um motorista utiliza o seu veículo, o que pode trazer benefícios relacionados à eficiência energética, infraestrutura das cidades, seguridade automotiva, planejamento urbano e etc. Neste artigo será abordado o atual estado do uso de técnicas de Aprendizado de Máquina (AM) para a definição de perfis de utilização de veículos e também, proposta uma abordagem para essa forma de descoberta de perfis de motoristas.

O restante deste artigo está organizado da seguinte forma. Em seguida, a Seção XXXXX aborda os trabalhos relacionados à descoberta de perfis de utilização de automóveis, mostrando os aspectos gerais e também aspectos mais específicos dos principais artigos. A Seção XXXXX propõe uma abordagem de solução do problema em questão e faz uma breve contextualização de suas características com o que foi abordado na seção XXXXXX. Por fim, a Seção XXXXX aborda os resultados que se quer atingir com a abordagem proposta.

\section{Organização do trabalho}

Nesta seção deve ser apresentado como está organizado o trabalho, sendo descrito, portanto, do que trata cada capítulo.
	
	% Capitulo 2: Segundo capítulo (arquivo Includes/Capitulo2.tex)
	% Capítulo 2
\chapter{Capítulo 2}

Este é o primeiro capítulo da parte central do trabalho, isto é, o desenvolvimento, a parte mais extensa de todo o trabalho. Geralmente o desenvolvimento é dividido em capítulos, cada um com subseções e subseções, cujo tamanho e número de divisões variam em função da natureza do conteúdo do trabalho.

Em geral, a parte de desenvolvimento é subdividida em quatro subpartes:

\begin{itemize}
   \item \textit{contextualização ou definição do problema} -- consiste em descrever a situação ou o contexto geral referente ao assunto em questão, devem constar informações atualizadas visando a proporcionar maior consistência ao trabalho;
   \item \textit{referencial ou embasamento teórico} -- texto no qual se deve apresentar os aspectos teóricos, isto é, os conceitos utilizados e a definição dos mesmos; nesta parte faz-se a revisão de literatura sobre o assunto, resumindo-se os resultados de estudos feitos por outros autores, cujas obras citadas e consultadas devem constar nas referências;
   \item \textit{metodologia do trabalho ou procedimentos metodológicos} -- deve constar o instrumental, os métodos e as técnicas aplicados para a elaboração do trabalho;
   \item \textit{resultados} -- devem ser apresentados, de forma objetiva, precisa e clara, tanto os resultados positivos quanto os negativos que foram obtidos com o desenvolvimento do trabalho, sendo feita uma discussão que consiste na avaliação circunstanciada, na qual se estabelecem relações, deduções e generalizações.
\end{itemize}

É recomendável que o número total de páginas referente à parte de desenvolvimento não ultrapasse 60 (sessenta) páginas.

\section{Seção 1}

Teste de figura:

\begin{figure}[htb]
	\centering
  	\includegraphics[scale=0.75]{Imagens/FiguraTeste.png}
  	\textsf{\caption{Teste de uma figura em formato .png}}
  	\label{fig:FiguraTeste}
\end{figure}


\section{Seção 2}

Referenciamento da figura inserida na seção anterior: \ref{fig:FiguraTeste}


\section{Seção 3}

Seção 3


\section{Seção 4}

Seção 4
	
	% Capitulo 3: Terceiro capítulo (arquivo Includes/Capitulo3.tex)
	% Capítulo 3
\chapter{Capítulo 3}

Algumas regras devem ser observadas na redação da dissertação/tese: 

\begin{itemize}
   \item ser claro, preciso, direto, objetivo e conciso, utilizando frases curtas e evitando ordens inversas desnecessárias;
   \item construir períodos com no máximo duas ou três linhas, bem como parágrafos com cinco linhas cheias, em média, e no máximo oito (ou seja, não construir parágrafos e períodos muito longos, pois isso cansa o(s) leitor(es) e pode fazer com que ele(s) percam a linha de raciocínio desenvolvida);
   \item a simplicidade deve ser condição essencial do texto; a simplicidade do texto não implica necessariamente repetição de formas e frases desgastadas, uso exagerado de voz passiva (como \textit{será iniciado}, \textit{será realizado}), pobreza vocabular etc. Com palavras conhecidas de todos, é possível escrever de maneira original e criativa e produzir frases elegantes, variadas, fluentes e bem alinhavadas;
   \item adotar como norma a ordem direta, por ser aquela que conduz mais facilmente o leitor à essência do texto, dispensando detalhes irrelevantes e indo diretamente ao que interessa, sem ``rodeios'' (verborragias);
   \item não começar períodos ou parágrafos seguidos com a mesma palavra, nem usar repetidamente a mesma estrutura de frase;
   \item desprezar as longas descrições e relatar o fato no menor número possível de palavras;
   \item recorrer aos termos técnicos somente quando absolutamente indispensáveis e nesse caso colocar o seu significado entre parênteses (ou seja, não se deve admitir que todos os que lerão o trabalho já dispõem de algum conhecimento desenvolvido no mesmo);
   \item dispensar palavras e formas empoladas ou rebuscadas, que tentem transmitir ao leitor mera ideia de erudição (até mesmo às vezes ilusória);
   \item não perder de vista o universo vocabular do leitor, adotando a seguinte regra prática: \textit{nunca escrever o que não se diria};
   \item termos coloquiais ou de gíria devem ser usados com extrema parcimônia (ou mesmo nem serem utilizados) e apenas em casos muito especiais, para não darem ao leitor a ideia de vulgaridade e descaracterizar o trabalho;
   \item ser rigoroso na escolha das palavras do texto, desconfiando dos sinônimos perfeitos ou de termos que sirvam para todas as ocasiões; em geral, há uma palavra para definir uma situação;
   \item encadear o assunto de maneira suave e harmoniosa, evitando a criação de um texto onde os parágrafos se sucedem uns aos outros como compartimentos estanques, sem nenhuma fluência entre si;
   \item ter um extremo cuidado durante a redação do texto, principalmente com relação às regras gramaticais e ortográficas da língua; geralmente todo o texto é escrito na forma impessoal do verbo, não se utilizando, portanto, de termos em primeira pessoa, seja do plural ou do singular.
\end{itemize}

Continuação do texto.


\section{Seção 1}

Teste de tabela.

\begin{table}[!htb]
   \textsf{\caption{Tabela sem sentido.}}
   \centering
   \medskip
   \begin{tabular}{c|p{4cm}}
      \hline
      \textbf{Título Coluna 1} & \textbf{Título Coluna 2} \\
      \hline
      Texto curto & Texto mais extenso, que requer mais de uma linha \\
      \hline
      \label{tab:TabelaSemSentido}
   \end{tabular}
\end{table}


\section{Seção 2}

Seção 2


\subsection{Subseção 2.1}

Referência à tabela definida no início: \ref{tab:TabelaSemSentido}


\subsection{Subseção 2.2}

Texto a ser enumerado.

\begin{enumerate}
   \item Item 1
   \item Item 2, com nota explicativa\footnote{Nota explicativa}
   \item Item 3
\end{enumerate}


\section{Seção 3}

Texto antes de equação.

\begin{equation}
   x = y + z
\end{equation}

Texto depois de equação.
	
	% Capitulo 4: Quarto capítulo (arquivo Includes/Capitulo4.tex)
	% Capítulo 4
\chapter{Capítulo 4}

\section{Seção 1}

Teste para símbolo

\simb[$\lambda$ (algum símbolo)]{$\lambda$}


\section{Seção 2}

Teste para abreviatura 

\abrv[UFRN -- Universidade Federal do Rio Grande do Norte]{UFRN}

\abrv[DIMAp -- Departamento de Informática e Matemática Aplicada]{DIMAp}
	
	% Capitulo 5: Quinto capítulo (arquivo Includes/Capitulo5.tex)
	% Capítulo 5
\chapter{Capítulo 5}

\section{Seção 1}

Seção 1


\section{Seção 2}

Alguns exemplos de citação: 

Na tese de Doutorado de Paquete \cite{PaquetePhD}, discute-se sobre algoritmos de busca local estocásticos aplicados a problemas de Otimização Combinatória considerando múltiplos objetivos. Por sua vez, o trabalho de \cite{KnowlesBoundedLebesgue}, publicado nos anais do IEEE CEC de 2003, mostra uma técnica de arquivamento também empregada no desenvolvimento de algoritmos evolucionários multi-objetivo, trabalho esse posteriormente estendido para um capítulo de livro dos mesmos autores \cite{KnowlesBoundedPareto}. Por fim, no relatório técnico de \citeonline{Jaszkiewicz}, fala-se sobre um algoritmo genético híbrido para problemas multi-critério, enquanto no artigo de jornal de Lopez \textit{et al.} \cite{LopezPaqueteStu} trata-se do \textit{trade-off} entre algoritmos genéticos e metodologias de busca local, também aplicados no contexto multi-critério e relacionado de alguma forma ao trabalho de Jaszkiewicz (\citeyear{Jaszkiewicz}).

Outros exemplos relacionados encontram-se em \cite{Silberschatz} (livro), \cite{DB2XML} (referência da Web) e \cite{Angelo} (dissertação de Mestrado).

\subsection{Subseção 5.1}

Subseção 5.1


\subsection{Subseção 5.2}

Subsection 5.2


\section{Seção 3}

Seção 3
	
		
	% Consideracoes finais
	% Considerações finais
\chapter{Considerações finais}

As considerações finais formam a parte final (fechamento) do texto, sendo dito de forma resumida (1) o que foi desenvolvido no presente trabalho e quais os resultados do mesmo, (2) o que se pôde concluir após o desenvolvimento bem como as principais contribuições do trabalho, e (3) perspectivas para o desenvolvimento de trabalhos futuros, como listado nos exemplos de seção abaixo. O texto referente às considerações finais do autor deve salientar a extensão e os resultados da contribuição do trabalho e os argumentos utilizados estar baseados em dados comprovados e fundamentados nos resultados e na discussão do texto, contendo deduções lógicas correspondentes aos objetivos do trabalho, propostos inicialmente.


\section{Principais contribuições}

Texto.


\section{Limitações}

Texto.


\section{Trabalhos futuros}

Texto.
	
	% Bibliografia (arquivo Capitulos/Referencias.bib)
   
	\bibliography{Capitulos/Referencias}
    \bibliographystyle{abnt-alf}
	
	% Página em branco
	\newpage

\end{document}